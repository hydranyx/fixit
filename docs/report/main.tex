\documentclass[12pt]{article}
\usepackage{minted}
\usepackage{rotating}
\usepackage{float}
\usepackage{placeins}
\usepackage[margin = 0.7 in]{geometry}
\usepackage{listings}
\usepackage{color}
\usepackage{caption}
\usepackage{enumitem}
\usepackage{textcomp}
\usepackage{booktabs}
\usepackage{longtable}
\usepackage{pdfpages}
\usepackage{siunitx}
\usepackage[english]{babel}

\newcommand{\NB}[0]{\textbf{N.B.~}}
\newcommand{\tx}[1]{\texttt{#1}}


\title{
  System Report Progress
}
\author{
  J\"{o}rdis Hollander (s2956543)\\
}
\date{\today}
\begin{document}
\maketitle

\section{Problem (which task is to be supported? for which users?)}

Anyone who owns a piano for a sufficiently long period of time will know that it
requires maintenance. However, it is not always clear what kind of maintenance
is required. If the piano is manifesting a particular problem, from the symptoms
it may be difficult to determine the best course of action. For example, when a
piano sounds strange or out of tune it does not mean that a piano tuner is
required immediately. The condition may be caused by humidity, dust or simply
the environment the piano is placed in. The condition could also be more serious
then what could be resolved by a tuner, in which case it would require a
technician. In the worst case, the piano is not even
worth repairing relative to its cost. In that case the owner would need advice on
which movie producer is looking for a piano to demolish in their production.
The system will support a home owner being able to diagnose their piano's issue
and to determine their next steps.

\section{Expert (name and description)}
The expert consulted for the system is a piano technician, by the name of Frank
Hollander. He was educated by the conservatory in Arnhem in all matters piano.
He has his own company performing piano tuning, repair and, sales which has been
active for over 30 years now. He is regularly consulted to provide advice to
people facing problems with their pianos. His experience allows him to often
formulate solutions simply from the description of the problem. In more complex
situations he makes the diagnosis in person.

\section{Role of knowledge technology for the problem}
Having an expert assist with the diagnosis process will help in determining the
next step to resolving the problem. Fixes may be trivial (provided knowledge),
may require a tuner, or a technician. Encoding the knowledge of an expert in a
knowledge base, will assist in aiding many piano owners. Their process would be
to consult the knowledge base, answer questions and then determine a guiding
diagnosis and advice.


\section{The knowledge models: problem solving model, domain model, rule model
  (description and explanation)}
The model is going to be inference based. The outcome is going to be the
diagnosis and advice on resolving it. The advice may be a solution that could be
performed by the client, or to consult a tuner or technician.

\section{User interface, functionality (description, explanation, justification)}
\section{Walkthrough of a session (explaining the internal and external working
  of the system)}

\section{Validation of knowledge models (test sessions with expert)}
\section{Task division among group members}
My situation is slightly unorthodox as was initially part of another group for
the majority of the project. In that group I constructed the a basic inference
engine in Rust, a command line application, handled web hosting, and constructed
a basic knowledge base.\\

As of my transition to my own group I have handled communication with a new
expert on my own topic, started on the web templates for interacting with the
user, refined the inference engine to more properly incorporate backward- and
forward-chaining and constructed a Dockerfile to ease deployment.\\

However given my topic change I still need to finish constructing the first
model for the new topic and construct the knowledge base. This will be done
early in the vacation followed by validation. There are still 2 primary bugs in
the inference engine that still need to be resolved and will be dealt with when
the rest of the system is in-place. The web view still needs to be finalized but
most of interface will completed this week.

\section{Reflection: experiences? role of knowledge technology? problems?
  evaluation of used techniques? lessons?}
The reflection process will occur whenever the project nears completion. However
it has so far been a fun exercise in communication and in programming practice.

\end{document}
